\documentclass[8pt]{article}
\usepackage{array, xcolor, lipsum, bibentry}
\usepackage[margin=2.25cm]{geometry}
\usepackage[utf8]{inputenc}
\usepackage{ragged2e}
 
\title{\bfseries\Huge Alberto García García}
\author{agarcia@dtic.ua.es}
\date{\today}
 
\definecolor{lightgray}{gray}{0.8}
\newcolumntype{L}{>{\raggedleft}p{0.14\textwidth}}
\newcolumntype{R}{p{0.8\textwidth}}
\newcommand\VRule{\color{lightgray}\vrule width 0.5pt}
 
\begin{document}
\begin{center}
	\Huge Alberto García-García\\
	%\large \texttt{< agarcia @ dtic.ua.es >}\\
	\Large \textbf{www.dtic.ua.es/$\sim$agarcia}\\
	\today
\end{center}
%\noindent\makebox[\linewidth]{\rule{0.88\paperwidth}{0.4pt}}
%\\
\bigskip
\begin{minipage}[ht]{0.65\textwidth}
%Carretera San Vicente del Raspeig s/n - 03690\\
Department of Computer Technology (DTIC)\\
University of Alicante (\textbf{Spain})\\
\end{minipage}
\hfill
\begin{minipage}[ht]{0.3\textwidth}
Phone: (+34) 645 71 65 31\\
\textbf{Mail: agarcia@dtic.ua.es}\\
%Mail: albert.garcia.ua@gmail.com\\
\end{minipage}
 
\section*{Interests and Objectives}

My main interest is \emph{High Performance Computing}. I am currently active in areas like \emph{Parallel Computing} and \emph{General-Purpose Computation on Graphics Processing Units (GPGPU)}. My motivation lies on applying all that computing knowledge to solve cross-disciplinary problems. In fact, my current projects involve parallel computing and 3D computer vision problems. My other related passions are physics, scientific simulations, 3D computer graphics, robotics and AI.

Being conscious of my limited knowledge in all those fields, I strive for seizing all the opportunities to learn and improve my skills in those fields and also discover current and future challenges to find my true passion and contribute where I can.
 
\section*{Work/Research Experience}
\begin{tabular}{L!{\VRule}R}
	2015-Today & {\bf Technician/Research Assistant}\\
	& \textbf{Department of Computer Technology, University of Alicante}\\
	& Working on the SIRMAVED national project (DPI2013-40534-R) which aims to promote the health and welfare of society from the design, development and evaluation of a novel therapy of cognitive rehabilitation for people with acquired brain injury or dependent people. This therapy will be based on the design and use of an intelligent and active monitoring environment system and a social autonomous robot providing interactive stimulation at home.\\
	&\\
2015 (Summer)&{\bf PRACE Summer Of High Performance Computing Student}\\
& \textbf{Jülich Supercomputing Centre, Forschungszentrum Jülich}\\
& Worked on parallel computing on GPUs. The goal was to accelerate, using CUDA, certain parts of the Fast Multipole Method (FMM) used to speed up the calculation of long-range forces in the $N$-body problem. The work was supervised by Ivo Kabadshow and Andreas Beckmann. Jiri Kraus, from NVIDIA, offered valuable support for the development.\\
& \\
2014--2015&{\bf Research Internship}\\
& \textbf{Department of Computer Technology, University of Alicante}\\
& Worked on 3D vision algorithms related to object recognition under time constraints using technologies like Kinect 2.0 and CUDA with the Jetson TK1 platform. The research was performed under the direction of José García-Rodríguez and Sergio Orts-Escolano.\\
& \\
2013--2014&{\bf Research Internship}\\
& \textbf{Department of Computer Technology, University of Alicante}\\
& Worked on computer vision and computational geometry algorithms. Our efforts were directed towards the development of an accelerated variant of the Iterative Closest Point method. The research was supervised by Higinio Mora-Mora and Jerónimo Mora-Pascual.\\
\end{tabular}
 
\section*{Educational Background}
\begin{tabular}{L!{\VRule}R}
2015--2016 &\textbf{Master's Degree in Automation and Robotics}\\
(Expected) & \textbf{University of Alicante}\\
& Master's Thesis: TBD.\\
& Supervisors: José García-Rodríguez.\\
& \\
2011--2015 &\textbf{Bachelor's Degree in Computer Engineering}\\
& \textbf{University of Alicante}\\
& High Academic Performance Group -- Average grade : 9.75/10\\
& Bachelor's Thesis (with Honors): "\textit{Towards a real-time 3D object recognition pipeline on mobile GPGPU computing platforms using low-cost RGB-D sensors}".\\
& Supervisors: José García-Rodríguez and Sergio Orts-Escolano\\
& \\
2014 & \textbf{Erasmus Intensive Programme: Big Data}\\
& \textbf{The University of Salford}\\
\end{tabular}

\section*{Honors and Awards}
\begin{tabular}{L!{\VRule}R}
2015 & \textbf{Degree in Computer Engineering Extraordinary Award} \\
& Awarded with the Degree Extraordinary Award for achieving the best academic record of the Degree in Computer Engineering (University of Alicante, 2011-2015) with an average grade of 9.75 over 10 points.\\
& \\
2014 & \textbf{CUDA Programming Contest}\\
& Winner of the CUDA Programming Contest organized by the Department of Computer Technology (DTIC) during the IV Workshop of scientific applications and computer vision on graphics processors (JGPU14) for the work entitled "\textit{CUDA Implementation of Kohonen maps}".
\end{tabular}

\section*{Grants}
\begin{tabular}{L!{\VRule}R}
2014 & \textbf{Research Collaboration Grant}\\
& Research collaboration grant for initiation in research tasks during an eight month stay (November, 2014 -- June, 2015) at the Department of Computer Technology at the University of Alicante funded by the \textit{Ministerio de Educación, Cultura y Deporte} (MECD) from Spain.
\end{tabular}

\section*{Research}

I am currently an intern at the \textit{Industrial Informatics and Computer Networks (I2RC)} research group which belongs to the \textit{Department of Computer Technology (DTIC)} of the University of Alicante. The research collaboration is being supervised by the professor José García-Rodríguez.

\subsection*{Projects participation}

\begin{tabular}{L!{\VRule}R}
	2014 -- Today & \textbf{SIRMAVED} (DPI2013-40534-R)\\
	& \textit{"Development of a comprehensive robotic system for monitoring and interaction for people with acquired brain damage and dependent people."} Spanish National Project. \\
\end{tabular}

\subsection*{Societies/Memberships}

\begin{itemize}
	\item \textbf{HiPEAC}: Member of the European Network of Excellence on High Performance and Embedded Architecture and Compilation.
	\item \textbf{AERFAI}: Member of the Spanish Association on Pattern Recognition and Image Analysis.
\end{itemize}

\subsection*{Publications}

\subsubsection*{Journals}

\begin{tabular}{L!{\VRule}R}
  \emph{\textbf{[j3]}} 2015&\textbf{Towards a real-time 3D object recognition pipeline on mobile GPGPU computing platforms using low-cost RGB-D sensors}. Alberto Garcia-Garcia, Sergio Orts-Escolano, Jose Garcia-Rodriguez, Miguel Cazorla. Journal of Real-Time Image Processing. \emph{(Under review)}.\\
  & \\ 
	\emph{\textbf{[j2]}} 2015&\textbf{3D Model Reconstruction using Neural Gas Accelerated on GPUs}. Sergio Orts-Escolano, Jose Garcia-Rodriguez, Jose Antonio Serra-Perez, Antonio Jimeno, Vicente Morell-Gimenez, Miguel Cazorla, Alberto Garcia-Garcia. Applied Soft Computing Journal. \textit{doi:10.1016/j.asoc.2015.03.042}. JCR IMPACT FACTOR: 2.140\\
	& \\
	\emph{\textbf{[j1]}} 2015&\textbf{3D Surface Reconstruction of noisy Point Clouds using Growing Neural Gas: Object/Scene Reconstruction}. Sergio Orts-Escolano, Jose Garcia-Rodriguez, Vicente Morell-Gimenez, Miguel Cazorla, Jose Antonio Serra-Perez, Alberto Garcia-Garcia. Neural Processing Letters. \textit{doi:10.1007/s11063-015-9421-x}. JCR IMPACT FACTOR: 2.005 \\
\end{tabular}

\subsubsection*{Conferences}

\begin{tabular}{L!{\VRule}R}
	\textit{\textbf{[c3]}} 2015&\textbf{Efficient Matching for the Iterative Closest Point Algorithm by using Low Cost Distance Metrics.} Higinio Mora-Mora, Jeronimo Mora-Pascual, Pablo Martinez-Gonzalez, Alberto Garcia-Garcia. International Conference on Applied Mathematics and Computational Methods in Engineering 2015.\\
		& \\
	\textit{\textbf{[c2]}} 2015&\textbf{Optimized Representation of 3D Sequences using Neural Networks}. Sergio Orts-Escolano, Jose Garcia-Rodriguez, Vicente Morell, Miguel Cazorla, Alberto Garcia-Garcia, Sergiu Ovidiu-Oprea. International Work-conference on the Interplay between Natural and Artificial Computation 2015.\\
	& \\
	\textit{\textbf{[c1]}} 2014&\textbf{Convergence Analysis and Validation of low Cost Distance Metrics for Computational Cost Reduction of the Iterative Closest Point algorithm}. Higinio Mora-Mora, Jerónimo Mora-Pascual, Pablo Martínez-González, Alberto García-García. Mathematical Modelling in Engineering \& Human Behaviour 2014.\\
\end{tabular}

\section*{Courses and training}

\begin{itemize}
	\item \textbf{Summer course on scientific applications and computer vision on graphics processors}\\ at the University of Alicante, 2013.
	\item \textbf{Workshop on scientific applications and computer vision on graphics processors}\\ at the University of Alicante, 2013.	
	\item \textbf{Interactive 3D Graphics (Highest Distinction)}\\
	Online at Udacity, 2013.
	\item \textbf{Functional Programming Principles in Scala (Highest Distinction)}\\
	Online at Coursera, 2013
\end{itemize}

\section*{Languages}
\begin{tabular}{L!{\VRule}R}
{\bf English}&{\bf Fluent (B2+)}\\
{Spanish}&{Native}\\
\end{tabular}

\section*{Reference List}
\begin{itemize}
  \item {\textbf{Ivo Kabadshow}\\
    Forschungszentrum Jülich\\
    Institute for Advanced Simulation\\
    Jülich Supercomputing Centre\\
    Wilhem-Johnen-Strasse\\
    52425 Jülich\\
    Germany\\
    +49 2461 61-8714\\
    i.kabadshow@fz-juelich.de\\}
  \item {\textbf{Andreas Beckmann}\\
    Forschungszentrum Jülich\\
    Institute for Advanced Simulation\\
    Jülich Supercomputing Centre\\
    Wilhem-Johnen-Strasse\\
    52425 Jülich\\
    Germany\\
    +49 2461 61-8713\\
    a.beckmann@fz-juelich.de\\}
  \item {\textbf{Jose Garcia-Rodriguez}\\
    University of Alicante\\
    Department of Computer Technology and Computation\\
    Carretera San Vicente del Raspeig s/n\\
    03690, San Vicente del Raspeig\\
    Spain\\
    +34 965 90 34 00-2616\\
    jgarcia@dtic.ua.es\\}

\end{itemize}
 
\end{document}
