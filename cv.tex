\documentclass[8pt]{article}
\usepackage{array, xcolor, lipsum, bibentry}
\usepackage[margin=2.25cm]{geometry}
\usepackage[utf8]{inputenc}
\usepackage{ragged2e}
\usepackage{hyperref}
\usepackage{url}
 
\title{\bfseries\Huge Alberto García García}
\author{agarcia@dtic.ua.es}
\date{\today}
 
\definecolor{lightgray}{gray}{0.8}
\newcolumntype{L}{>{\raggedleft}p{0.14\textwidth}}
\newcolumntype{R}{p{0.8\textwidth}}
\newcommand\VRule{\color{lightgray}\vrule width 0.5pt}
 
\begin{document}
\begin{center}
	\Huge Alberto García-García\\
	%\large \texttt{< agarcia @ dtic.ua.es >}\\
	\Large PhD Student\\
	\Large \textbf{\url{www.dtic.ua.es/~agarcia}}\\
	\today
\end{center}
%\noindent\makebox[\linewidth]{\rule{0.88\paperwidth}{0.4pt}}
%\\
\bigskip
\begin{minipage}[ht]{0.65\textwidth}
%Carretera San Vicente del Raspeig s/n - 03690\\
Department of Computer Technology (DTIC)\\
University of Alicante (\textbf{Spain})\\
\end{minipage}
\hfill
\begin{minipage}[ht]{0.3\textwidth}
Phone: (+34) 645 71 65 31\\
\textbf{Mail: agarcia@dtic.ua.es}\\
%Mail: albert.garcia.ua@gmail.com\\
\end{minipage}
 
\section*{Interests and Objectives}

My main area of interest lies on the intersection of 3D Computer Vision, Machine Learning, and GPU Programming. My Thesis is focused on real-time 3D scene semantic understanding by using deep learning techniques. Other related passions are physics, scientific simulations, and 3D computer graphics.
 
\section*{Work/Research Experience}
\begin{tabular}{L!{\VRule}R}
	2017 & {\bf Research Intern}\\
	(Ongoing) & \textbf{Oculus Research by Facebook Inc., Redmond}\\
	& Working with the Machine Perception team under the supervision of Richard Newcombe.\\
	&\\
	2016  & {\bf{Machine Learning Software Intern}}\\
	(4 months) & \textbf{NVIDIA Corporation, Santa Clara Headquarters}\\
															& Worked on efficient background removal or foreground segmentation using deep architectures with Caffe and also on synthetic dataset augmentation techniques for self-driving cars. I was embedded in the Mobile Visual Computing group at NVIDIA Research lead by Jan Kautz, and the Camera/Algorithms team lead by Howard Waterfall. The internship was mentored by Shalini De Mello from NVIDIA Research.\\
	&\\
	2015-2016 & {\bf Technician/Research Assistant}\\
	(8 months) & \textbf{Department of Computer Technology, University of Alicante}\\
	& Worked on the SIRMAVED national project (DPI2013-40534-R). We developed a C++/Qt rehabilitation application, for people with acquired brain damage, using multiple sensors: Kinect V2, Leap Motion, and Tobii Eye X.\\
	&\\
2015 &{\bf PRACE Summer Of High Performance Computing Student}\\
(2 months) & \textbf{Jülich Supercomputing Centre, Forschungszentrum Jülich}\\
& Worked on parallel computing on GPUs. We accelerated with CUDA certain parts of the Fast Multipole Method (FMM) used to speed up the calculation of long-range forces in the $N$-body problem. The work was supervised by Ivo Kabadshow and Andreas Beckmann. Jiri Kraus, from NVIDIA, offered valuable support for the development.\\
& \\
2014--2015&{\bf Research Internship}\\
(8 months) & \textbf{Department of Computer Technology, University of Alicante}\\
& Worked on 3D vision algorithms related to object recognition under time constraints using technologies like Kinect 2.0 and CUDA with the Jetson TK1 platform. The research was performed under the direction of José García-Rodríguez and Sergio Orts-Escolano.\\
& \\
2013--2014&{\bf Research Internship}\\
(4 months) & \textbf{Department of Computer Technology, University of Alicante}\\
& Worked on computer vision and computational geometry algorithms. Our efforts were directed towards the development of an accelerated variant of the Iterative Closest Point method. The research was supervised by Higinio Mora-Mora and Jerónimo Mora-Pascual.\\
\end{tabular}
 
\section*{Educational Background}
\begin{tabular}{L!{\VRule}R}
2016--2020 &\textbf{Doctor of Philosophy in Machine Learning and Computer Vision}\\
& \textbf{University of Alicante}\\
& PhD Thesis: TBD\\
& Advisors: José García-Rodríguez and Sergio Orts-Escolano\\
& \\
2015--2016 &\textbf{Master's Degree in Automation and Robotics}\\ 
& \textbf{University of Alicante}\\
& Average Grade: 9.86/10\\
& Master's Thesis: \emph{3D Object Recognition with Convolutional Neural Networks}\\
& Advisors: José García-Rodríguez and Jorge Pomares-Baeza\\
& \\
2011--2015 &\textbf{Bachelor's Degree in Computer Engineering}\\
& \textbf{University of Alicante}\\
& High Academic Performance Group -- Average grade : 9.75/10\\
& Bachelor's Thesis (with Honors): \emph{Towards a real-time 3D object recognition pipeline on mobile GPGPU computing platforms using low-cost RGB-D sensors}.\\
& Advisors: José García-Rodríguez and Sergio Orts-Escolano\\
& \\
2014 & \textbf{Erasmus Intensive Programme: Big Data}\\
& \textbf{The University of Salford}\\
\end{tabular}

\section*{Honors and Awards}
\begin{tabular}{L!{\VRule}R}
2016 & \textbf{Master's Degree in Automation and Robotics Extraordinary Award} \\
& Awarded for achieving the best academic record of the Master in Automation and Robotics (University of Alicante, 2015-2016) with an average grade of 9.86 out of 10 points. \\
& \\
2015 & \textbf{Best Academic Record in Technology Degrees Award} \\
& Awarded for achieving the best academic record among the students of all technology-related degrees (University of Alicante, 2011-2015). \\
& \\
2015 & \textbf{Bachelor's Degree in Computer Engineering Extraordinary Award} \\
& Awarded for achieving the best academic record of the Degree in Computer Engineering (University of Alicante, 2011-2015) with an average grade of 9.75 out of 10 points.\\
& \\
2015 & \textbf{Summer of HPC Ambassador Award} \\
& Awarded with the Best HPC Ambassador Award of the Summer of High Performance Computing programme by PRACE for the outreach achieved communicating the results of the programme.\\
& \\
2014 & \textbf{CUDA Programming Contest Winner}\\
			 & Winner of the CUDA Programming Contest organized by the University of Alicante, for the work entitled \emph{CUDA Implementation of Kohonen maps}.
\end{tabular}

\section*{Grants}
\begin{tabular}{L!{\VRule}R}
2016 & \textbf{FPU Grant for PhD Studies}\\
& Granted by the Spanish Ministerio de Economía y Competitividad de España (MINECO). Considered to be the most competitive grant funded by the Spanish government with only 37 grants awarded for the Computer Science area (nationwide).\\
& \\
2015 & \textbf{Research Initiation Grant}\\
& Research collaboration grant to foster the initiation in research tasks to gain knowledge about current scientific and technical problems and possible solution methods during a six month stay (January, 2016 -- June, 2016) at the Department of Computer Technology at the University of Alicante co-funded by the \textit{Industrial Informatics and Computer Networks (I2RC)} research group and the \textit{Vicerrectorado de Investigación, Desarrollo e Innovación} of the \textit{University of Alicante}.\\
& \\
2014 & \textbf{Research Collaboration Grant}\\
& Research collaboration grant for initiation in research tasks during an eight month stay (November, 2014 -- June, 2015) at the Department of Computer Technology at the University of Alicante funded by the \textit{Ministerio de Educación, Cultura y Deporte} (MECD) from Spain.
\end{tabular}

\section*{Research}

I am currently a PhD Student at the \textit{Industrial Informatics and Computer Networks (I2RC)} research group which belongs to the \textit{Department of Computer Technology (DTIC)} of the University of Alicante. I am also part of the \textit{3D Perception Lab} formed at that University to unify efforts from different departments and researchers who share focus on 3D data.

\subsection*{Projects Participation}

\begin{tabular}{L!{\VRule}R}
	2016 -- Today & \textbf{COMBAHO} (Spanish National Project TIN2016-76515-R)\\
	& \textit{"System for enhancing autonomy of people with acquired brain injury and dependent on their integration into society."} \\
	& \\
	2015 & \textbf{SPPEXA/GROMEX} (German Priority Programme 1648)\\
	& \textit{"Priority Programme Software for Exascale Computing."/"Unified Long-range Electrostatics and Dynamic Protonation for Realistic Biomolecular Simulations on the Exascale"} \\
	& \\
	2014 -- 2015 & \textbf{SIRMAVED} (Spanish National Project DPI2013-40534-R)\\
	& \textit{"Development of a comprehensive robotic system for monitoring and interaction for people with acquired brain damage and dependent people."} \\
\end{tabular}

\subsection*{Publications}

\subsubsection*{Journals}

\begin{tabular}{L!{\VRule}R}
	\emph{\textbf{[j13]}} & \textbf{A Review On Deep Learning Techniques Applied to Semantic Segmentation}. Alberto Garcia-Garcia, Sergio Orts-Escolano, Sergiu Oprea, Victor Villena-Martinez, Jose Garcia-Rodriguez. Transactions on pattern Analysis and Machine Intelligence (2017). \emph{(Under review)}. Pre-print version available at {\href{https://arxiv.org/abs/1704.06857}{arXiv}}.\\
	& \\
	\emph{\textbf{\href{http://www.sciencedirect.com/science/article/pii/S1077314217301182}{[j12]}}} & \textbf{A Study of the Effect of Noise and Occlusion on the Accuracy of Convolutional Neural Networks applied to 3D Object Recognition}. Alberto Garcia-Garcia, Jose Garcia-Rodriguez, Francisco Gomez-Donoso, Miguel Cazorla. Computer Vision and Image Understanding (2017). \emph{doi:10.1016/j.cviu.2017.06.006}.\\
	& \\
\emph{\textbf{\href{http://www.sciencedirect.com/science/article/pii/S0957417416306819}{[j11]}}} & \textbf{Automatic selection of molecular descriptors using random forest: Application to drug discovery}. Gaspar Cano, Jose Garcia-Rodriguez, Alberto Garcia-Garcia, Horacio Perez-Sanchez, Jón Atli Benediktsson, Anil Thapa, Alastair Barr. Expert Systems (2017). \emph{doi:10.1016/j.eswa.2016.12.008}.\\
	& \\
\emph{\textbf{\href{http://www.mdpi.com/1424-8220/17/2/243/htm}{[j10]}}} & \textbf{A Quantitative Comparison of Calibration Methods for RGB-D Sensors Using Different Technologies}. Víctor Villena-Martínez, Andrés Fuster-Guilló, Jorge Azorín-López, Marcelo Saval-Calvo, Jeronimo Mora-Pascual, Jose Garcia-Rodriguez, Alberto Garcia-Garcia. Sensors (2017). \emph{doi:10.3390/s17020243}.\\
	& \\
	\emph{\textbf{\href{http://journals.plos.org/plosone/article?id=10.1371/journal.pone.0164694}{[j9]}}} & \textbf{Computational Analysis of Distance Operators for the Iterative Closest Point Algorithm}. Higinio Mora-Mora, Jeronimo Mora-Pascual, Alberto Garcia-Garcia, Pablo Martinez-Gonzalez. PLOSOne (2016). \emph{doi:10.1371/journal.pone.0164694}.\\
	& \\
	\emph{\textbf{\href{http://onlinelibrary.wiley.com/doi/10.1111/exsy.12160/abstract}{[j8]}}} & \textbf{Automatic Schaeffer's Gestures Recognition System}. Francisco Gomez‐Donoso, Miguel Cazorla, Alberto Garcia‐Garcia, Jose Garcia‐Rodriguez. Expert Systems (2016). \emph{doi:10.1111/exsy.12160}.\\
	& \\
	\emph{\textbf{\href{http://link.springer.com/article/10.1007/s00521-016-2258-z}{[j7]}}} & \textbf{Evaluation of Sampling Method Effects in 3D Non-rigid Registration}. Marcelo Saval-Calvo, Jorge Azorin-Lopez, Andres Fuster-Guillo, Jose Garcia-Rodriguez, Sergio Orts-Escolano, Alberto Garcia-Garcia. Neural Computing and Applications (2016). \emph{doi:10.1007/s00521-016-2258-z}.\\
	& \\
	\emph{\textbf{\href{http://link.springer.com/article/10.1007/s00521-016-2224-9}{[j6]}}} & \textbf{Multi-sensor 3D Object Dataset for Object Recognition with Full Pose Estimation}. Alberto Garcia-Garcia, Sergio Orts-Escolano, Sergiu Oprea, Jose Garcia-Rodriguez, Jorge Azorin-Lopez, Marcelo Saval-Calvo, Miguel Cazorla. Neural Computing and Applications (2016). \emph{doi:10.1007/s00521-016-2224-9}.\\
	& \\
	\emph{\textbf{\href{http://link.springer.com/article/10.1007/s00521-016-2585-0}{[j5]}}} & \textbf{Bioinspired Point Cloud Representation: 3D Object Tracking}. Jose Garcia-Rodriguez, Sergio Orts-Escolano, Miguel Cazorla, Vicente Morell, Jorge Azorin, Marcelo Saval, Alberto Garcia-Garcia. Neural Computing and Applications (2016). \emph{doi:10.1007/s00521-016-2585-0}.\\
	& \\
	\emph{\textbf{\href{http://www.sciencedirect.com/science/article/pii/S0213131516000067}{[j4]}}} & \textbf{Drug Solubility Prediction with Support Vector Machines on Graphic Processor Units}. Gaspar Cano, Jose Garcia-Rodriguez, Sergio Orts-Escolano, Alberto Garcia-Garcia, Jorge Peña-Garcia, Alfonso Perez-Garrido, Horacio Perez-Sanchez. Revista Internacional de Métodos Numéricos para Cálculo y Diseño en Ingeniería (2016). \emph{doi:10.1016/j.rimni.2015.12.001}.\\
	& \\
	\emph{\textbf{\href{http://link.springer.com/article/10.1007/s11554-016-0607-x}{[j3]}}} &\textbf{Interactive 3D object recognition pipeline on mobile GPGPU computing platforms using low-cost RGB-D sensors}. Alberto Garcia-Garcia, Sergio Orts-Escolano, Jose Garcia-Rodriguez, Miguel Cazorla. Journal of Real-Time Image Processing (2016). \emph{doi:10.1007/s11554-016-0607-x}.\\
\end{tabular}

\clearpage

\subsubsection*{Journals (cont.)}

\begin{tabular}{L!{\VRule}R}
	\emph{\textbf{\href{http://www.sciencedirect.com/science/article/pii/S1568494615002008}{[j2]}}} &\textbf{3D Model Reconstruction using Neural Gas Accelerated on GPUs}. Sergio Orts-Escolano, Jose Garcia-Rodriguez, Jose Antonio Serra-Perez, Antonio Jimeno, Vicente Morell-Gimenez, Miguel Cazorla, Alberto Garcia-Garcia. Applied Soft Computing Journal (2015). \emph{doi:10.1016/j.asoc.2015.03.042}.\\
	& \\
	\emph{\textbf{\href{http://link.springer.com/article/10.1007/s11063-015-9421-x}{[j1]}}} &\textbf{3D Surface Reconstruction of noisy Point Clouds using Growing Neural Gas: Object/Scene Reconstruction}. Sergio Orts-Escolano, Jose Garcia-Rodriguez, Vicente Morell-Gimenez, Miguel Cazorla, Jose Antonio Serra-Perez, Alberto Garcia-Garcia. Neural Processing Letters (2015). \emph{doi:10.1007/s11063-015-9421-x}. \\
\end{tabular}

\subsubsection*{Conferences and Congresses}

\begin{tabular}{L!{\VRule}R}
	\textit{\textbf{\href{http://ieeexplore.ieee.org/abstract/document/7965883/}{[c8]}}} &\textbf{LONCHANet: A Slice-based CNN Architecture for Real-time 3D Object Recognition.} Francisco Gomez-Donoso, Alberto Garcia-Garcia, Jose Garcia-Rodriguez, Sergio Orts-Escolano, Miguel Cazorla. International Joint Conference on Neural Networks (IJCNN 2017).\\
	& \\
	\textit{\textbf{\href{http://ieeexplore.ieee.org/abstract/document/7965885/}{[c7]}}} &\textbf{A Recurrent Neural Network Based Schaeffer Gesture Recognition System.} Sergiu Oprea, Alberto Garcia-Garcia, Jose Garcia-Rodriguez, Sergio Orts-Escolano, Miguel Cazorla. International Joint Conference on Neural Networks (IJCNN 2017) (Accepted).\\
	& \\
	\textit{\textbf{\href{http://on-demand.gputechconf.com/gtc/2016/presentation/s6286-albert-garcia-towards-a-unified-cpu-gpu-codebase.pdf}{[c6]}}} &\textbf{Towards a Unified CPU/GPU Codebase for Linear Scaling FMM Coulomb Solver.} Alberto Garcia-Garcia, Ivo Kabadshow, Andreas Beckmann. GPU Technology Conference (GTC 2016).\\
	& \\
	\textit{\textbf{\href{http://ieeexplore.ieee.org/document/7727386/}{[c5]}}} &\textbf{PointNet: A 3D Convolutional Neural Network for Real-Time Object Class Recognition.} Alberto Garcia-Garcia, Francisco Gomez-Donoso, Jose Garcia-Rodriguez, Sergio Orts-Escolano, Miguel Cazorla, Jorge Azorin-Lopez. International Joint Conference on Neural Networks (IJCNN 2016).\\
	& \\
	\textit{\textbf{\href{http://link.springer.com/chapter/10.1007/978-3-319-40528-5_22}{[c4]}}} &\textbf{Accelerating an FMM-based Coulomb Solver with GPUs.} Alberto Garcia-Garcia, Ivo Kabadshow, Andreas Beckmann. SPPEXA Symposium proceedings, Lecture Notes in Computational Science and Engineering (LNCSE 2016).\\
	& \\
	\textit{\textbf{\href{http://link.springer.com/chapter/10.1007/978-3-319-18833-1_27}{[c3]}}} &\textbf{Optimized Representation of 3D Sequences using Neural Networks}. Sergio Orts-Escolano, Jose Garcia-Rodriguez, Vicente Morell, Miguel Cazorla, Alberto Garcia-Garcia, Sergiu Ovidiu-Oprea. International Work-conference on the Interplay between Natural and Artificial Computation (IWINAC 2015).\\
	& \\
	\textit{\textbf{\href{http://www.inase.org/library/2015/barcelona/bypaper/AMCME/AMCME-06.pdf}{[c2]}}} &\textbf{Efficient Matching for the Iterative Closest Point Algorithm by using Low Cost Distance Metrics.} Higinio Mora-Mora, Jeronimo Mora-Pascual, Pablo Martinez-Gonzalez, Alberto Garcia-Garcia. International Conference on Applied Mathematics and Computational Methods in Engineering (AMCME 2015).\\
	& \\
	\textit{\textbf{\href{http://jornadas.imm.upv.es/Modelling2014}{[c1]}}} &\textbf{Convergence Analysis and Validation of low Cost Distance Metrics for Computational Cost Reduction of the Iterative Closest Point algorithm}. Higinio Mora-Mora, Jerónimo Mora-Pascual, Pablo Martínez-González, Alberto García-García. Mathematical Modelling in Engineering \& Human Behaviour (MMEHB 2014).\\
\end{tabular}

\subsubsection*{Poster Presentations}

\begin{tabular}{L!{\VRule}R}
	\textit{\textbf{\href{http://www.gputechconf.com/resources/poster-gallery/2017/deep-learning-artificial-intelligence}{[p2]}}} &\textbf{LonchaNet: A Slice-based CNN Architecture for Real-time 3D Object Recognition.} Alberto Garcia-Garcia, Francisco Gomez-Donoso, Miguel Cazorla, Sergio Orts-Escolano, Jose Garcia-Rodriguez. GPU Technology Conference (GTC 2017).\\
	& \\
	\textit{\href{http://www.gputechconf.com/resources/poster-gallery/2016/algorithms}{\textbf{[p1]}}} &\textbf{One Kernel To Rule Them All. Performance-Portable FMM for CPUs and GPUs.} Ivo Kabadshow, Andreas Beckmann, Alberto Garcia-Garcia. GPU Technology Conference (GTC 2016).\\
\end{tabular}

\subsection*{Societies/Memberships}

\begin{itemize}
	\item \textbf{IV\&L Net}: Member of the European Network on Integrating Vision and Language, ICT COST Action IC1307.
	\item \textbf{HiPEAC}: Member of the European Network of Excellence on High Performance and Embedded Architecture and Compilation.
	\item \textbf{AERFAI}: Member of the Spanish Association on Pattern Recognition and Image Analysis.
	\item \textbf{DTIC}: Member of the Council of the Department of Computer Technology, University of Alicante (Student representative).
\end{itemize}


\section*{Courses and training}

\begin{itemize}
	\item \textbf{Self-Driving Car Nanodegree}\\ Udacity, (Ongoing) 2017.
	\item \textbf{iV\&L Net Training School 2016: 2nd Summer School on Integrating Vision and Language}\\ University of Malta, 2016.
	\item \textbf{Deep Learning}\\ Online at Udacity, 2016.
	\item \textbf{Machine Learning (Ongoing)}\\ Online at Coursera, 2016.
	\item \textbf{Summer Course on Scientific Applications and Computer Vision on Graphics Processors}\\ at the University of Alicante, 2013.
	\item \textbf{Workshop on Scientific Applications and Computer Vision on Graphics Processors}\\ at the University of Alicante, 2013.	
	\item \textbf{Interactive 3D Graphics (Highest Distinction)}\\
	Online at Udacity, 2013.
	\item \textbf{Functional Programming Principles in Scala (Highest Distinction)}\\
	Online at Coursera, 2013.
\end{itemize}

\section*{Languages}
\begin{tabular}{L!{\VRule}R}
{\bf English}&{\bf Fluent (B2+)}\\
{Spanish}&{Native}\\
\end{tabular}

\clearpage

\section*{Reference List}
\begin{itemize}
	\item {\textbf{Jose Garcia-Rodriguez} (jgarcia@dtic.ua.es)\\
    University of Alicante\\
    Department of Computer Technology and Computation\\
    Spain\\}
	\item {\textbf{Sergio Orts-Escolano} (sorts@dtic.ua.es)\\
University of Alicante\\
		Department of Computer Science and Artificial Intelligence\\
		(Former Microsoft Research)\\
		Spain\\}

  \item {\textbf{Ivo Kabadshow} (i.kabadshow@fz-juelich.de)\\
    Forschungszentrum Jülich\\
    Institute for Advanced Simulation\\
    Jülich Supercomputing Centre\\
    Germany\\}

  \item {\textbf{Howard Waterfall} (hwaterfall@nvidia.com)\\
	NVIDIA Corporation\\
	Camera/Algorithms\\
	United States of America\\}

  \item{\textbf{Shalini De Mello} (shalinig@nvidia.com)\\
	NVIDIA Corporation\\
	Mobile Visual Computing (NVIDIA Research)\\
	United States of America\\}

\end{itemize}
 
\end{document}
